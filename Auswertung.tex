
\newpage
\section{Auswertung}

    \subsection{Grundlagen Switchkonfiguration}
    
    Die Switchkonfiguration zur Bildung von VLANs beinhaltet die Einstellungen "`tagged"' (markiert) und "`untagged"' (nicht markiert). 
    "`Tagged"' wird für Uplink-Ports verwendet, um Datenpakete mit VLAN-Tags zu versehen und die Kommunikation zwischen VLANs über mehrere 
    Switches hinweg zu ermöglichen. "`Untagged"' wird bei Ports genutzt, die direkt mit Endgeräten verbunden sind und den Datenverkehr innerhalb 
    eines VLANs ohne VLAN-Tags übertragen. Die richtige Konfiguration dieser Einstellungen ist wichtig, um VLANs effektiv zu nutzen und den Datenverkehr 
    zu isolieren.
    
    \subsection{Vergleich und Beurteilung}
    \textbf{Subnetting:}
    \\Subnetting ist eine Methode zur Aufteilung eines Netzwerks in kleinere Teilnetze. Es basiert auf der Aufteilung des IP-Adressraums und der 
    Verwendung von Subnetzmasken. Subnetting allein bietet jedoch keine eigentliche Sicherheit oder Zugriffskontrolle zwischen den Teilnetzen. Es ermöglicht lediglich die logische Aufteilung 
    des Netzwerks in verschiedene Subnetze, um die IP-Kommunikation zu organisieren.
    \\\\\textbf{VLANs:}
    \\VLANs ermöglichen die logische Aufteilung eines Netzwerks unabhängig von der physischen Infrastruktur. Durch die Verwendung von VLAN-IDs können Netzwerkgeräte in verschiedene virtuelle 
    Netzwerke segmentiert werden. VLANs bieten die Möglichkeit, den Datenverkehr zwischen den VLANs zu isolieren und Zugriffssteuerungen anzuwenden.
    \\\\Zusammenfassend lässt sich sagen, dass VLANs im Vergleich zu Subnetting eine höhere Sicherheit bieten, da sie die Trennung und Steuerung des Datenverkehrs zwischen virtuellen Netzwerken 
    ermöglichen. Dennoch sollten zusätzliche Sicherheitsmechanismen in Verbindung mit VLANs implementiert werden, um ein robustes Sicherheitsniveau zu gewährleisten.

