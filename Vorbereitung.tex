\newpage
\section{Vorbereitung}

    \subsection{Begriffserklärung}

        \subsubsection{Switch-Stacking}

        Switch-Stacking ist ein wichtiges Merkmal von Netzwerk-Switches.
        Diese Switches können miteinander verbunden werden, um als 
        logische Einheit zu fungieren. Durch das Zusammenschalten weiterer Switche, wird die Netzwerkkapazität 
        dank der höheren Anzahl verfügbarer Ports, besserer Ausfallsicherheit und der Möglichkeit 
        Link-Aggregation zu betreiben, erheblich erhöht. 
        Switch-Stacking wird nur von stapelbaren Switches unterstützt. \\\\
        Switches in einem Stack können mittels DAC-Kabeln, optischen Transceiver oder Stacking-Kabeln verbunden werden. 
        Es gibt einen Stack-Master, der das Zentrum des Stack-Systems ist und die Konfigurationsdaten verwaltet. 
        Die anderen Switches im Stack werden als Stack-Slaves bezeichnet. 
        Der Stack-Master kann von Benutzern verwaltet werden, und falls er ausfällt, wird ein neuer Master-Switch unter den Slaves ausgewählt.\\\\
        Zusammenfassend lassen sich folgende Vorteile von Switch-Stacking erschließen:
        \begin{itemize}
            \item Verbesserung der Zuverlässigkeit und Flexibilität des Netzwerkes
            \item Erhöhung der Bandbreite und Vereinfachung der Vernetzung
            \item hohe Skalierbarkeit des Netzwerkes
        \end{itemize}

        \subsubsection{Switchkaskadierung}

        Kaskadierung ist die traditionelle Methode zum Verbinden mehrerer Ethernet-Switches 
        und umfasst verschiedene Methoden für unterschiedliche Netzwerktopologien.\\
        Durch die Verknüpfung mehrerer Switches können Benutzer mehrere Ports haben, 
        die jeden Switch miteinander verbinden, unabhängig voneinander konfiguriert 
        und als Gruppe verwaltet werden können. 
        In einem Kaskaden-Switch-Netzwerk sind Daisy-Chain-Topologie und 
        Sterntopologie zwei gängige Methoden. 

        \subsubsection{Spanning-Tree-Verfahren}

        Der Spanning-Tree-Algorithmus führt eine Reihe von Schritten aus, 
        um sicherzustellen, dass die Topologie schleifenfrei ist und das Ethernet ordnungsgemäß funktioniert:\\
        \begin{enumerate}
            \item Root-Bridge-Auswahl – Zuerst wählt STP eine Root-Bridge aus. Dies ist der wichtigste Schalter in der Topologie. Es ist die Wurzel des azyklischen Baums.
            \item Schleifentopologie-Erkennung – Sobald die Root-Bridge ausgewählt ist, beginnt sie mit dem Senden von Spanning-Tree-Nachrichten (BPDUs). Der Switch verwendet diese Nachrichten, um den Teil der Topologie zu finden, der die Schleife enthält.
            \item Bestimmen der Port-Rollen – Nach dem Bestimmen des Loop-Teils der Topologie platziert jeder Switch so viele Switch-Ports wie nötig, um sicherzustellen, dass die Topologie schleifenfrei ist.
            \item Dropout – Switches tauschen weiterhin Nachrichten aus, um die Verfügbarkeit von Links und Nachbarkontakten zu verfolgen. Wenn die Verbindung oder der Switch ausfällt, führt der Switch die Schritte 2 und 3 erneut aus, um sicherzustellen, dass die neue Topologie schleifenfrei ist.
        \end{enumerate}

        \subsubsection{Auto-Negoatiation}

        Auto-Negotiation ist eine Funktion, die es zwei Netzwerken mit unterschiedlichen Geschwindigkeiten
        ermöglicht, zu kommunizieren und sich an eine Geschwindigkeit anpasst, 
        die für beide Netzwerke geeignet ist. 
        Beispielsweise verfügt ein Switch über einen 1-Gbit/s-Port (Gigabit-Ethernet), 
        der mit einem 100-Mbit/s-Port (Fast Ethernet) an einem anderen Switch verbunden ist. 
        Die Portgeschwindigkeiten an beiden Enden müssen gleich sein, um eine Verbindung herzustellen. 
        Das Autonegotiation-Protokoll teilt Baudrate, Duplexmodusstatus und Flusssteuerungsinformationen zwischen zwei Ports. 
        Sobald der Port die obigen Parameterinformationen empfangen hat, 
        passt er seinen Pegel entsprechend den Fähigkeiten des Peer-Ports an.

        \subsubsection{AutoUplink(MDI/MDI-X)}

        Ein Ethernet-Netzwerkport (z. B. an einem Switch) verwendet die automatische Uplink-Funktion, um zu erkennen, 
        an welchen Switch er senden und empfangen (MDI/MDIX) soll. 
        Mit der automatischen Uplink-Funktion können sowohl Crossover-Kabel 
        als auch 1:1-Netzwerkkabel verwendet werden.

        \newpage
        \subsubsection{Link Aggregation/ Port Trunking}

        \newpage
        \subsubsection{Vollduplex-Betrieb}

        \newpage
        \subsubsection{Mac-Adressenfilter}

    \newpage
    \subsection{Netzwerktrennung}

        \subsubsection{VLAN}

        \newpage
        \subsubsection{VLAN-Tagging}

    \newpage
    \subsection{Übersicht des NETGEAR-Switches}